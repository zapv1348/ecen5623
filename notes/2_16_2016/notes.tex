\documentclass{article}

\usepackage[margin=1in]{geometry}
\usepackage{graphicx}
\usepackage{amsmath}
\usepackage{amsfonts}


\author{Zachary Vogel}
\date{\today}
\title{Notes in ECEN 5623}

\begin{document}
\maketitle


\section*{STuff}
Homework set 2 due 2/22
excercise 3 due 3/3
2/25 quiz.\\

\section*{Dynamic Priority}
Power PC architecture is pretty common.\\

RM= bunch of blocks fitting in a suitcase.\\
EDF/LLF = bunch of playdoy in a suitcase.\\

EDF/LLF bound proof.\\
RM gets used when you are more concerned about safety.\\

best effort and RM most common.\\

4 approaches for software in embedded systems
\begin{enumerate}
    \item no software, all hardware or higher someone else
    \item write all your own.
    \item Reuse somebody elses code, buy a commercial or open RTOS
    \item Use a non-RTOS and make it work like an RTOS
\end{enumerate}

book: theory, then practical.\\

EDF can be bad because you don't know what is going to happen when you fail without intense timing analysis.\\
harder to debug.\\
because of this, EDF isn't used as much.\\
EDF doesn't deal with unpredictability as well as RM.\\
SCHED\_DEADLINE is a EDF scheduler for linux.\\


least laxity.\\
even more difficult to implement because you have to guess how much execution time is left.\\
overload is difficult.\\

tend not to be used because of the additional risk.\\

If a system is going to fail and explode anyway, wouldn't EDF or LLF be worth it?\\

For a lightly loaded system (i.e. lots of slack time), RM will generally give the same result as EDF/LLF.\\

The more harmonic something is, the more likely RM, LLF, and EDF are to be the same.\\

next time: applications of RM theory.


\end{document}
